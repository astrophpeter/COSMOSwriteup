% mnras_template.tex
%
% LaTeX template for creating an MNRAS paper


%%%%%%%%%%%%%%%%%%%%%%%%%%%%%%%%%%%%%%%%%%%%%%%%%%
\documentclass[a4paper,fleqn,usenatbib]{mnras}
\usepackage{newtxtext,newtxmath}
% Depending on your LaTeX fonts installation, you might get better results with one of these:
%\usepackage{mathptmx}
%\usepackage{txfonts}
\usepackage[T1]{fontenc}
\usepackage{ae,aecompl}


%%%%% AUTHORS - PLACE YOUR OWN PACKAGES HERE %%%%%

\usepackage{graphicx}
\usepackage{amsmath}	
\usepackage{amssymb}

%%%%%%%%%%%%%%%%%%%%%%%%%%%%%%%%%%%%%%%%%%%%%%%%%%

%%%%%%%%%%%%%%%%%%% TITLE PAGE %%%%%%%%%%%%%%%%%%%

% Title of the paper, and the short title which is used in the headers.
% Keep the title short and informative.
\title[COSMOS SFHs]{Galaxy Zoo: Star Formation Histories in the COSMOS Survey}

% The list of authors, and the short list which is used in the headers.
% If you need two or more lines of authors, add an extra line using \newauthor
\author[]{
,$^{1}$\thanks{}
,$^{2}$
$^{2,3}$
$^{3}$
\\
% List of institutions
$^{1}$
$^{2}$
$^{3}$
}

% These dates will be filled out by the publisher
\date{Accepted XXX. Received YYY; in original form ZZZ}

% Enter the current year, for the copyright statements etc.
\pubyear{2015}

% Don't change these lines
\begin{document}
\label{firstpage}
\pagerange{\pageref{firstpage}--\pageref{lastpage}}
\maketitle

% Abstract of the paper
\begin{abstract}


\end{abstract}

% Select between one and six entries from the list of approved keywords.
% Don't make up new ones.
\begin{keywords}
keyword1 -- keyword2 -- keyword3
\end{keywords}

%%%%%%%%%%%%%%%%%%%%%%%%%%%%%%%%%%%%%%%%%%%%%%%%%%

%%%%%%%%%%%%%%%%% BODY OF PAPER %%%%%%%%%%%%%%%%%%

\section{Introduction}
Becky's group environment paper \cite{smethurst2017galaxy}.
\section{Data}
   
   \subsection{Multi-wavelength data}
   
   In this study we used the K$_{s}$-selected catalog of the COSMOS/UltraVISTA field from \cite{muzzin2013public}.  
   The catalog contains PSF-matched photometry in 30 photometric bands covering the wavelength range 0.15$\micron$ 
   $\rightarrow$ 24$\micron$ and includes the available $GALEX$ \citep{martin2005}, CFHT/Subaru \citep{capak2007}, 
   UltraVISTA \citep{mcCraken2012}, S-COSMOS \citep{sanders2007}, and zCOSMOS \citep{lilly2009} datasets.

   We required the rest-frame U - V and V - J colours which were calculated using the \textsc{eazy} code 
   \citep{eazycode}. \textsc{eazy} determines rest-frame colors by integrating the best-fit SED through the redshifted filter curves
over the appropriate wavelength range.For the U and V filters the Johnson response curves calculated in \cite{maiz2006} 
   were used. The J filter used the 2MASS response curve from \cite{2mass2006}.
 
   \subsection{Environment data}
  
   A continuous environmental density estimate ($\delta$) for each galaxy was obtained from \cite{darvish2015}.
   The method used to estimate environment was weighted Voronoi Tessellation (WVT) which is an iteration 
   on the simple Voronoi Tessellation technique \citep{ebeling1993,bernardeau1996}. In simple Voroni 
   Tesselation, galaxies are divided into redshift planes to estimate local density. WVT uses a Monte-Carlo 
   acceptance-rejection process to incorporate a weighted contribution from each galaxy in the density
   estimate. The weight assigned to each galaxy corresponds to the likelihood of it residing in the 
   particular redshift slice being examined. Weighting the contributions aims to reduce the impact of 
   redshift uncertainties on the density estimates.

   \cite{darvish2015} showed that WVT outperforms several other density estimation methods when evaluated 
   against known surface density profiles. Moreover, unlike other density estimators (nearest neighbour methods) 
   WVT makes no underlying assumptions about the geometry and morphology of the structures in the density field. 
   (However it is noted that Veronoi Tessellation can not be used to assign density estimates to galaxies near the
   edge of the field.) Overall \cite{darvish2015} concludes that WVT is a robust density estimator. 

   The WVT algorithm and a comparison between several density estimation methods can be found 
   in Sections 4.3 and 6 of \cite{darvish2015} respectively. In this study galaxies were binned into low
   ($\log(1+\delta)\leq-0.25$), intermediate ($-0.25 <\log(1+\delta)< 0.25$), and high ($\log(1+\delta)\geq0.25$)
   density groups.

   \subsection{Galaxy Zoo Hubble Morphological classifications}
   
   Morphological classifications of galaxies were obtained from the Galaxy Zoo Hubble\footnote{\url{https://hubble.galaxyzoo.org}} (GZH) citizen 
   science project \citep{galaxyzooHubble}. GZH allowed several independent visual classifications of each galaxy image by volunteers, the question flowchart for each
   image is shown in Figure 4 of \cite{galaxyzooHubble}. Volunteers were shown colour-composite images of real and simulated \textit{Hubble Space Telescope} galaxies to
   correct for classification redshift bias. 

   Following \cite{smethurst2015galaxy} we exploit the continuous nature of GZH vote fractions (see Section 3 of
   \cite{galaxyzooHubble}) to investigate the role morphology has in populations of galaxies. This is acheived by weighting
   each galaxies contribition to the analysis of a population by one of its morphological vote fractions. An example of a galaxy
   with associated vote fractions of disc ($p_{d}$) and smooth ($p_{s}$) is shown in the top right of Figure [FIG].

   The GZH project provides detailed classifications of 119,849 galaxies from publicly-released \textit{Hubble
   Space Telescope Legacy} programs conducted with the Advanced Camera for Surveys. In this study we used a subset of the total GZH sample corresponding to 84,954
   galaxies found in the COSMOS Survey \citep{scoville2007,koekemoer2007}.

\section{Methods}   

   \subsection{Modelling Star Formation History}

   We used the publically available, Markov Chain Monte Carlo based \citep{mackey2013},
   \textsc{starpy}\footnote{\url{http://github.com/zooniverse/starpy/}} code to infer the star formation 
   history (SFH) of each galaxy in our sample. Inputs of redshift and of rest-frame U - V and V - J colours, 
   combined with the stellar population model of \cite{bruzual2003}, a solar metallicity, and a Chabrier IMF 
   \cite{chabrier2003} produces a SFH for each galaxy. These models do not account for intrinsic dust. An 
   explanation of how \textsc{starpy} works can be found in Section 3

   The SFH model used is parameterised by $t_{q}$ and $\tau$ [Gyrs], where $t_{q}$ is the onset time of quenching
   and $\tau$ characterises the rate of quenching. Increaing $\tau$ represents an exponetially slower quench. 
   Assuming that all galaxies formed at $t=0$ [Gyrs] with an initial burst of constant SFR ($\text{SFR}_{0}$), 
   we write the SFH over cosmic time ($0\leq t \text{ }[\text{Gyrs}]\leq 13.8$) as:

 \begin{equation}
       \text{SFR}(t) = \begin{cases}
                 \text{SFR}_{0}(t_{q}) & t \leq t_{q} \\
                 \text{SFR}_{0}(t_{q})\exp\bigg[-\frac{(t-t_{q})}{\tau}\bigg] & t > t_{q} 
                \end{cases}
        \label{eq:model}
 \end{equation}

   

   \begin{equation}
      P(\theta_{k}) = \begin{cases}
                       1 & 0\leq t_{q}\text{ [Gyr] } \leq 13.8 \text{ and }  0 \leq \tau\text{ [Gyr] } \leq 4 \\
                       0 & \text{otherwise}
                      \end{cases}
      \label{eq:prior}
   \end{equation}

\section{Conclusions}

\section*{Acknowledgements}


%%%%%%%%%%%%%%%%%%%%%%%%%%%%%%%%%%%%%%%%%%%%%%%%%%

%%%%%%%%%%%%%%%%%%%% REFERENCES %%%%%%%%%%%%%%%%%%


\bibliographystyle{mnras}
\bibliography{refs} % if your bibtex file is called example.bib


%%%%%%%%%%%%%%%%%%%%%%%%%%%%%%%%%%%%%%%%%%%%%%%%%%

%%%%%%%%%%%%%%%%% APPENDICES %%%%%%%%%%%%%%%%%%%%%

\appendix


%%%%%%%%%%%%%%%%%%%%%%%%%%%%%%%%%%%%%%%%%%%%%%%%%%


% Don't change these lines
\bsp	% typesetting comment
\label{lastpage}
\end{document}

% End of mnras_template.tex
