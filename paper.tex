% mnras_template.tex
%
% LaTeX template for creating an MNRAS paper
%
% v3.0 released 14 May 2015
% (version numbers match those of mnras.cls)
%
% Copyright (C) Royal Astronomical Society 2015
% Authors:
% Keith T. Smith (Royal Astronomical Society)

% Change log
%
% v3.0 May 2015
%    Renamed to match the new package name
%    Version number matches mnras.cls
%    A few minor tweaks to wording
% v1.0 September 2013
%    Beta testing only - never publicly released
%    First version: a simple (ish) template for creating an MNRAS paper

%%%%%%%%%%%%%%%%%%%%%%%%%%%%%%%%%%%%%%%%%%%%%%%%%%
% Basic setup. Most papers should leave these options alone.
\documentclass[a4paper,fleqn,usenatbib]{mnras}

% MNRAS is set in Times font. If you don't have this installed (most LaTeX
% installations will be fine) or prefer the old Computer Modern fonts, comment
% out the following line
\usepackage{newtxtext,newtxmath}
% Depending on your LaTeX fonts installation, you might get better results with one of these:
%\usepackage{mathptmx}
%\usepackage{txfonts}

% Use vector fonts, so it zooms properly in on-screen viewing software
% Don't change these lines unless you know what you are doing
\usepackage[T1]{fontenc}
\usepackage{ae,aecompl}


%%%%% AUTHORS - PLACE YOUR OWN PACKAGES HERE %%%%%

% Only include extra packages if you really need them. Common packages are:
\usepackage{graphicx}	% Including figure files
\usepackage{amsmath}	% Advanced maths commands
\usepackage{amssymb}	% Extra maths symbols

%%%%%%%%%%%%%%%%%%%%%%%%%%%%%%%%%%%%%%%%%%%%%%%%%%

%%%%% AUTHORS - PLACE YOUR OWN COMMANDS HERE %%%%%

% Please keep new commands to a minimum, and use \newcommand not \def to avoid
% overwriting existing commands. Example:
%\newcommand{\pcm}{\,cm$^{-2}$}	% per cm-squared

%%%%%%%%%%%%%%%%%%%%%%%%%%%%%%%%%%%%%%%%%%%%%%%%%%

%%%%%%%%%%%%%%%%%%% TITLE PAGE %%%%%%%%%%%%%%%%%%%

% Title of the paper, and the short title which is used in the headers.
% Keep the title short and informative.
\title[COSMOS SFHs]{Galaxy Zoo: Star Formation Histories in the COSMOS Survey}

% The list of authors, and the short list which is used in the headers.
% If you need two or more lines of authors, add an extra line using \newauthor
\author[]{
,$^{1}$\thanks{}
,$^{2}$
$^{2,3}$
$^{3}$
\\
% List of institutions
$^{1}$
$^{2}$
$^{3}$
}

% These dates will be filled out by the publisher
\date{Accepted XXX. Received YYY; in original form ZZZ}

% Enter the current year, for the copyright statements etc.
\pubyear{2015}

% Don't change these lines
\begin{document}
\label{firstpage}
\pagerange{\pageref{firstpage}--\pageref{lastpage}}
\maketitle

% Abstract of the paper
\begin{abstract}


\end{abstract}

% Select between one and six entries from the list of approved keywords.
% Don't make up new ones.
\begin{keywords}
keyword1 -- keyword2 -- keyword3
\end{keywords}

%%%%%%%%%%%%%%%%%%%%%%%%%%%%%%%%%%%%%%%%%%%%%%%%%%

%%%%%%%%%%%%%%%%% BODY OF PAPER %%%%%%%%%%%%%%%%%%

\section{Introduction}

\section{Methods}

Starpy from becky's paper \cite{smethurst2015galaxy}.
Becky's group environment paper \cite{smethurst2017galaxy}.
UltraVista catalogue paper \cite{muzzin2013public}
\section{Data}
   
   \subsection{Multi-wavelength data}
   
   This study is based on a K$_{s}$-selected catalog of the COSMOS/UltraVISTA field from \cite{muzzin2013public}.  
   The catalog contains PSF-matched photometry in 30 photometric bands covering the wavelength range 0.15$\micron$ 
   $\rightarrow$ 24$\micron$ and includes the available $GALEX$ \citep{martin2005}, CFHT/Subaru \citep{capak2007}, 
   UltraVISTA \citep{mcCraken2012}, S-COSMOS \citep{sanders2007}, and zCOSMOS \citep{lilly2009} datasets.

   \subsection{Environment data}
   
   Environment data from \cite{darvish2015}.

   \begin{itemize}
      \item Method used is Weighted Voronoi Tessellation
      \item  Quote from darvish: Unlike the nearest neighbor, Voronoi tessellation is scale-independent and is able to span a wide range of physical lengths. Also, it
does not make any assumptions about the geometry and
morphology of the structures in the density field. This
characteristic makes it superior to adaptive kernel and
nearest neighbor methods.

       \item  Quote from Darvish: However, this comes at the expense
of a computationally expensive process by
making several Monte-Carlo samples. Apart from
its computational time, it is a robust estimator.

      \item formula: \begin{equation}\Sigma(r_{i})=\frac{1}{A_{i}}\end{equation}
   \end{itemize}


   \subsection{Galaxy Zoo Hubble Morphological classifications}
    
   Galaxy zoo hubble data paper \cite{galaxyzooHubble}
   
\section{Model}

   \begin{equation}
      \text{SFR}(t) = \begin{cases}
                \text{SFR}_{0}(t_{q}) & t \leq t_{q} \\
                \text{SFR}_{0}(t_{q})\exp\bigg[-\frac{(t-t_{q})}{\tau}\bigg] & t > t_{q} 
               \end{cases}
       \label{eq:model}
   \end{equation}

\section{Probalistic Fitting}


   \begin{equation}
      P(\theta_{k}) = \begin{cases}
                       1 & 0\leq t_{q}\text{ [Gyr] } \leq 13.8 \text{ and }  0 \leq \tau\text{ [Gyr] } \leq 4 \\
                       0 & \text{otherwise}
                      \end{cases}
      \label{eq:prior}
   \end{equation}

\section{Conclusions}

\section*{Acknowledgements}


%%%%%%%%%%%%%%%%%%%%%%%%%%%%%%%%%%%%%%%%%%%%%%%%%%

%%%%%%%%%%%%%%%%%%%% REFERENCES %%%%%%%%%%%%%%%%%%


\bibliographystyle{mnras}
\bibliography{refs} % if your bibtex file is called example.bib


%%%%%%%%%%%%%%%%%%%%%%%%%%%%%%%%%%%%%%%%%%%%%%%%%%

%%%%%%%%%%%%%%%%% APPENDICES %%%%%%%%%%%%%%%%%%%%%

\appendix


%%%%%%%%%%%%%%%%%%%%%%%%%%%%%%%%%%%%%%%%%%%%%%%%%%


% Don't change these lines
\bsp	% typesetting comment
\label{lastpage}
\end{document}

% End of mnras_template.tex
